% Esse texto é propriedade de
% Agosto de 2012
% Author: Douglas Fabiano de Sousa Nunes 
% Instituto Federal de Educação, Ciência e Tecnologia do Sul de Minas Gerais - IFSULDEMINAS
%
% 

% Para impressão é só substituir
%\documentclass[handout]{beamer}
\documentclass{beamer}

% Setup appearance:
\usetheme{Darmstadt}
\usefonttheme[onlylarge]{structurebold}
\setbeamerfont*{frametitle}{size=\normalsize,series=\bfseries}
\setbeamertemplate{navigation symbols}{}

% Standard packages
\usepackage[english,brazil]{babel}
\usepackage[utf8]{inputenc}
\usepackage{times}
\usepackage[T1]{fontenc}
%para usar figuras e cores no texto
\usepackage{graphicx,color}


% Setup TikZ
\usepackage{tikz}
\usetikzlibrary{arrows}
\tikzstyle{block}=[draw opacity=0.7,line width=1.4cm]

% Title
\title{Sistemas Operacionais II}
\author{Douglas Fabiano de Sousa Nunes}
\institute{Instituto Federal de Educação, Ciência e Tecnologia do Sul de Minas Gerais - IFSULDEMINAS}
\date{Curso Técnico em Informática - Módulo II, 2012}

% The main document
\begin{document}

\begin{frame}
  \titlepage
  \begin{figure}
   \centering
   \includegraphics[width=1cm]{../../../imagens/logotipo}
  \end{figure}
\end{frame}

\begin{frame}[t]{Sumário}
  \tableofcontents
\end{frame}

\section{Introdução}
  \begin{frame}
    \begin{center}
      \Large{\textcolor{green}{\textbf{INTRODUÇÃO}}}
    \end{center}
  \end{frame}
  \subsection{O UNIX e sua Origem} 
   \begin{frame}{UNIX}
    Para começarmos a entender o que é UNIX preciso voltar alguns anos no tempo\dots \textcolor{blue}{cerca de 50-60!}
    \begin{columns}
      \begin{column}{5cm}
	\begin{figure}
	  \centering
	  \includegraphics[width=5cm]{../../../imagens/primeiro_computador}
	\end{figure}
      \end{column}
      \begin{column}{6cm}
	\begin{itemize}
	  \uncover<2->
	  {\item Os computadores eram enormes \textcolor{blue}{(dimensões de uma casa e, às vezes, de um estádio de futebol)}}
	  \uncover<3->
	  {\item Cada fabricante tinha o seu próprio Sistema Operacional}
	  \uncover<4->
	  {\item Os \textit{softwares} eram de propósito específicos}
	\end{itemize}
      \end{column}
    \end{columns}
   \end{frame} 
   
   \begin{frame}[t]{UNIX}
    Foi na tentativa de mudar essa realidade \textbf{[especificidade dos \textit{softwares}]} que surgiu o UNIX.
    \begin{figure}
      \centering
      \includegraphics[scale=0.25]{../../../imagens/pdp-7}
      \caption{PDP-7, a UNIX Genesis Machine, e os criadores do UNIX}
    \end{figure}
   \end{frame}
  
\section{A Filosofia do UNIX}
  \begin{frame}
     \begin{center}
        \Large{\textcolor{green}{\textbf{A FILOSOFIA DO UNIX}}}
     \end{center}
   \end{frame}
  \subsection{Conhecendo o UNIX}    
   \begin{frame}[t]{A Filosofia do UNIX}
      De acordo com \cite{GM1995} a filosofia UNIX é formada por um conjunto de princípios:
      \begin{itemize}
	% A linha abaixo adiciona um espaço no beamer
	\vspace{\baselineskip}
	\uncover<2->
	{\item Pequeno é belo}
	\uncover<3->
	{\item Construa programas que fazem uma coisa bem feita}
	\uncover<4->
	{\item Disponibilize uma versão protótipo o mais breve possível}
	\uncover<5->
	{\item Prefira portabilidade a eficiência}
	\uncover<6->
	{\item Armazene dados em arquivos simples, como arquivos do tipo texto}
	\uncover<7->
	{\item Use as funcionalidades disponíveis do sistema para tirar vantagem}
	\uncover<8->
	{\item Evite fazer programas engessados}
	\uncover<9->
	{\item Etc\dots}
      \end{itemize}
   \end{frame}

   \begin{frame}[t]{A Filosofia do UNIX}
      Segundo o ponto de vista de \cite{RES2003} as regras da filosofia UNIX obedecem aos seguintes princípios: 
      \begin{itemize}
	% A linha abaixo adiciona um espaço no beamer
	\vspace{\baselineskip}
	\uncover<2->
	{\item \textbf{Regra da Modularidade:}\uncover<3-> { escreva peças simples conectadas por interfaces simples}}
	\uncover<4->
	{\item \textbf{Regra da Clareza:}\uncover<5-> { clareza é melhor que inteligência}}
	\uncover<6->
	{\item \textbf{Regra da Composição:}\uncover<7-> { construa programas para serem conectados com outros programas}}
	\uncover<8->
	{\item \textbf{Regra da Simplicidade:}\uncover<9-> { projete para simplicidade; adicione complexidade apenas onde é necessário}}
	\uncover<10->
	{\item \textbf{Regra da Transparência:}\uncover<11-> { projete de forma a dar visibilidade, para fazer do processo de depuração e inspeção \ 
							     uma tarefa fácil}}
      \end{itemize}
   \end{frame}

    \begin{frame}[t]{A Filosofia do UNIX}
      Segundo o ponto de vista de \cite{RES2003} as regras da filosofia UNIX obedecem aos seguintes princípios: 
      \begin{itemize}
	% A linha abaixo adiciona um espaço no beamer
	\vspace{\baselineskip}
	\uncover<2->
	{\item \textbf{Regra do Silêncio:}\uncover<3-> { quando um programa não tem nada surpreendente para dizer, ele não deve dizer nada}}
	\uncover<4->
	{\item \textbf{Regra do Reparo:}\uncover<5-> { quando é inevitável falhar, falhe ruidosamente e o mais cedo possível}}
	\uncover<6->
	{\item \textbf{Regra da Otimização:}\uncover<7-> { faça protótipo antes do refinamento. Coloque em funcionamento antes de ser otimizado}}
	\uncover<8->
	{\item \textbf{Regra da Diversidade:}\uncover<9-> { desconfie de todas as alegações sobre a existência de um ``modo correto de fazer as coisas''}}
      \end{itemize}
      % A linha abaixo adiciona um espaço no beamer
      \vspace{\baselineskip}
      \uncover<10->
      {\textcolor{blue}{Para conhecer outras regras, consulte \cite{RES2003}}}
   \end{frame}

   \begin{frame}{A Filosofia do UNIX}
      \centering
      \begin{figure}
	\includegraphics[scale=0.3]{../../../imagens/kiss}
	\caption{\textit{Kiss Principle} \cite{SW2010}} 
      \end{figure}
   \end{frame}

\section{Variantes do UNIX}
  \begin{frame}
     \begin{center}
        \Large{\textcolor{green}{\textbf{VARIANTES DO UNIX}}}
     \end{center}
   \end{frame}
  \subsection{Solaris, FreeBSD, Linux...}
   \begin{frame}[t]{Variantes do UNIX}
      Um consórcio da indústria que se formou em 1996, chamado \textbf{\textit{Open Group}}, propiciou o surgimento de diversas variantes do UNIX:
      \begin{itemize}
	\vspace{\baselineskip}
	\uncover<2->
        {\item Solaris da Sun Microsystems}
	\uncover<3->
	{\item A Família BSD}
	\begin{itemize}
	  \uncover<4->
	  {\item FreeBSD}
	  \uncover<5->
	  {\item NetBSD}
	  \uncover<6->
	  {\item OpenBSD}
	\end{itemize}
      \only<7>
      {\item \alert{e}}
      \uncover<8->
      {\item \textbf{Linux}}
      \end{itemize}
   \end{frame}

    \begin{frame}[t]{Variantes do UNIX}
      Dentre as variantes citadas, algumas estão de acordo com a especificação (mantida pela \textbf{\textit{Open Group}}) \
      e outras apenas se comportavam como um sistema UNIX, afinal, por dentro possuíam características específicas \\
      \vspace{\baselineskip}
      \uncover<2->
      {Neste sentido, existem dois grandes grupos de variantes UNIX:}
      \begin{itemize}
	\uncover<3->
        {\item Aqueles em acordo com a \textit{Single UNIX Specification}}
	\uncover<4->
	{\item Os clones}
	\begin{itemize}
	  \vspace{\baselineskip}
	  \uncover<5->
	  {\item Um programa clone possui funções similares a outro programa, porém tem um código-fonte totalmente diferente}
	\end{itemize}
      \end{itemize}
      \vspace{\baselineskip}
      \uncover<6->
      {\textbf{Unix-like:} termo utilizado para descrever Sistemas Operacionais descendentes dessas variantes}
   \end{frame}

   \begin{frame}[t]{Variantes do UNIX}
      \begin{figure}
	\includegraphics[scale=0.32]{../../../imagens/unix-like-family}
	\caption{Família Unix-like} 
      \end{figure}
   \end{frame}

\section{Linux}
  \begin{frame}
     \begin{center}
        \Large{\textcolor{green}{\textbf{LINUX}}}
     \end{center}
   \end{frame}
  \subsection{A História do Linux}
   \begin{frame}{Linux}
      \alert{Li}nux \alert{n}ão é \alert{U}NI\alert{X}!
      \centering
      \begin{figure}
	\includegraphics[scale=0.5]{../../../imagens/linustovarlds}
	\caption{\textbf{Linus Tovarlds:} o Criador do Linux} 
      \end{figure}
   \end{frame}

   \begin{frame}[t]{Linux}
      O \textbf{Linux} foi desenvolvido em conformidade com o POSIX, um padrão para a construção de Sistemas Operacionais criado para normatizar o UNIX \\
      \vspace{\baselineskip}
      \textcolor{blue}{Mas qual é a vantagem disso?}
      \begin{itemize}
	\vspace{\baselineskip}
	\uncover<2->
	{\item O Linux funciona em centenas de arquiteturas diferentes}
	\uncover<3->
	{\item É compatível com a maioria dos \textit{softwares} disponíveis para UNIX}
      \end{itemize}
      \uncover<4->
      {\begin{block}{A primeira versão do Linux}
	\centering
	A versão 0.01 do Linux tinha somente um desenvolvedor (\textcolor{blue}{advinha quem?}) e aproximadamente dez mil linhas de código na linguagem C. Os códigos \ 
        somavam alguns quilobytes (KB) 
      \end{block}}
   \end{frame}

   \begin{frame}[t]{Linux}
      O \textbf{Linux} foi desenvolvido em conformidade com o POSIX, um padrão para a construção de Sistemas Operacionais criado para normatizar o UNIX \\
      \vspace{\baselineskip}
      \textcolor{blue}{Mas qual é a vantagem disso?}
      \begin{itemize}
	\vspace{\baselineskip}
	\uncover<2->
	{\item O Linux funciona em centenas de arquiteturas diferentes}
	\uncover<3->
	{\item É compatível com a maioria dos \textit{softwares} disponíveis para UNIX}
      \end{itemize}
      \uncover<4->
      {\begin{exampleblock}{As versões atuais do Linux}
 	\centering
 	O Linux hoje conta com uma comunidade gigante de desenvolvedores e possui milhões de linhas de código. Os códigos \ 
         somam algumas centenas de megabytes (MB) 
      \end{exampleblock}}
   \end{frame}


    \begin{frame}[t]{Vantagens do Linux}
      Segundo \cite{SW2010}, as principais vantagens do Linux são: 
      \begin{itemize}
	% A linha abaixo adiciona um espaço no beamer
	\vspace{\baselineskip}
	\uncover<2->
	{\item \textbf{Gratuito:}\uncover<3-> { ser gratuito significa que o usuário não terá nenhum ônus para baixar e instalar o sistema completo}}
	\uncover<4->
	{\item \textbf{Comunidade:}\uncover<5-> { a comunidade do Linux é muito grande. É mais fácil você encontrar usando Linux do que qualquer outro \
						  sistema proprietário}}
	\uncover<6->
	{\item \textbf{Estável:}\uncover<7-> { o Linux foi feito para nunca parar. Algumas configurações podem ser ativadas sem a necessidade de reinicialização \
                                               do ambiente}}
	\uncover<8->
	{\item \textbf{Seguro:}\uncover<9-> { o modelo de segurança do Linux é amplamente reconhecido por sua robustez e qualidade comprovada}}
      \end{itemize}
   \end{frame}

    \begin{frame}[t]{Vantagens do Linux}
      Segundo \cite{SW2010}, as principais vantagens do Linux são: 
      \begin{itemize}
	% A linha abaixo adiciona um espaço no beamer
	\vspace{\baselineskip}
	\uncover<2->
	{\item \textbf{Portável:}\uncover<3-> { basta o fabricante realizar alguns ajustes no \textit{kernel} do Linux para funcionar em seu \
                                                \textit{hardware}}}
	\uncover<4->
	{\item \textbf{Escalável:}\uncover<5-> { a característica do Linux (\textit{kernel} + \textit{software}) faz com que ele funcione em um \
						 simples dispositivo móvel e até mesmo em supercomputadores}}
	\uncover<6->
	{\item \textbf{Antivírus:}\uncover<7-> { o sistema de permissões nativo do Linux é suficiente para protegê-lo de ameaças e vírus}}
	\uncover<8->
	{\item \textbf{Flexível:}\uncover<9-> { o Linux é o sistema mais flexível que existe. Como o seu código fonte é livre, as personalizações \
						são ``ilimitadas''}}
      \end{itemize}
   \end{frame}
 
\section{O Projeto GNU}
  \begin{frame}
     \begin{center}
        \Large{\textcolor{green}{\textbf{O PROJETO GNU}}}
     \end{center}
   \end{frame}
  \subsection{GNU is Not UNIX}
   \begin{frame}[t]{Projeto GNU}
      \centering
      O nome do projeto foi uma brincadeira irônica dos programadores da época. \\
      \vspace{\baselineskip}
      \uncover<2->
      {\textbf{GNU} é um acrônimo para "\alert{G}NU is \alert{N}ot \alert{U}NIX"}
      \uncover<3->
      {\begin{figure}
	\includegraphics[scale=0.2]{../../../imagens/simboloGNU}
	\caption{Símbolo do projeto GNU}
      \end{figure}}
      \uncover<4->
      {\begin{center}
	 \textcolor{blue}{Todo programa livre acompanhado do código-fonte era um GNU}
       \end{center}}
   \end{frame}

    \begin{frame}[t]{O Projeto GNU}
      A ideia do projeto GNU consistia em:
      \begin{itemize}
	% A linha abaixo adiciona um espaço no beamer
	\vspace{\baselineskip}
	\uncover<2->
	{\item Construir um Sistema Operacional como o UNIX, que oferecesse somente programas livres e que fossem suficientes para substituírem qualquer \
	       sistema não livre, no caso o próprio UNIX}
	\uncover<3->
	{\begin{itemize}
	 \item \alert{A partir da versão 7 o UNIX fechou o acesso ao código-fonte}
	\end{itemize}}
	\uncover<4->
	{\begin{figure}
	  \includegraphics[scale=0.37]{../../../imagens/stallman}
	  \caption{Richard M. Stallman: o criador do projeto GNU}
	\end{figure}}
      \end{itemize}
   \end{frame}
 
\section{Free Software Foundation}
  \begin{frame}
     \begin{center}
        \Large{\textcolor{green}{\textbf{FREE SOFTWARE FOUNDATION}}}
     \end{center}
   \end{frame}
  \subsection{Protegendo a liberdade}
    \begin{frame}[t]{\textit{Free Software Foundation}}
      \uncover<2->
      {A \textit{Free Software Foundation} (FSF) ou "Fundação para o \textit{Software} Livre", em português, trata-se de uma organização sem fins lucrativos \
      para promover e defender a liberdade de usuários de \textit{software} livre \\}
      \vspace{\baselineskip}
      \uncover<3->
      {A FSF foi criada também por \textbf{Richard M. Stallman} e teve como propósito angariar fundos para o financiamento de projetos, tal como o \
      projeto GNU \\}
      \vspace{\baselineskip}
      \uncover<4->
      {Para garantir as liberdades dos \textit{softwares}, a FSF precisava de um documento formal e, para tanto, foi criada uma licença chamada GNU GPL \\}
      \vspace{\baselineskip}
      \uncover<5->
      {\textcolor{blue}{Vejamos o que ela define\dots}}
   \end{frame}

    \begin{frame}[t]{\textit{Free Software Foundation}}
      A licença GNU GPL (\textit{General Public Lincense}), em linhas gerais, fornece quatro liberdades:
      \begin{enumerate}
	% A linha abaixo adiciona um espaço no beamer
	\vspace{\baselineskip}
	\uncover<2->
	{\item Liberdade de executar o programa para qualquer propósito}
	\uncover<3->
	{\item Liberdade de estudar como o programa funciona e adaptá-lo às suas necessidades}
	\uncover<4->
	{\item Liberdade de redistribuir cópias de modo que você possa ajudar o seu próximo}
	\uncover<5->
	{\item Liberdade de melhorar o programa e liberar os seus melhoramentos, de modo que toda a comunidade se beneficie deles}
      \end{enumerate}
   \end{frame}

\section{Referências}
  \subsection{Bibliografia}
  \begin{frame}[t]{Referências Bibliográficas}
    \bibliographystyle{abnt-alf}
    \bibliography{/home/douglas/Dropbox/Referencias/referencias}
  \end{frame}

\end{document}
